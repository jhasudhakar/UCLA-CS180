% Author: Zhehao Wang 404380075 zhehao@cs.ucla.edu

% Grammar package: http://tex.stackexchange.com/questions/24886/which-package-can-be-used-to-write-bnf-grammars

\documentclass{article}
\topmargin = 0in
\oddsidemargin = 0in
\evensidemargin = \oddsidemargin
\textwidth = 6.5in
\textheight = 8in
\usepackage{amsthm}
\usepackage{amsmath}
\usepackage{syntax}
\usepackage{graphicx}

\usepackage{algorithm}
\usepackage[noend]{algpseudocode}

\makeatletter
\def\BState{\State\hskip-\ALG@thistlm}
\makeatother

\title{CS180 Homework 8}
\author{Zhehao Wang 404380075 (Dis 1B)}
\date{May 24, 2016}

\begin{document}
\maketitle

\begin{description}

\item[1]{Min cut of the graph}
  
  \textbf{Algorithm description}: given the graph $G = (V, E)$, $V=v_1...v_n$; We first do $\frac{n}{2}$ MaxFlow between $(v_1, v_2), (v_3, v_4)...(v_{n-1}, v_n)$, then $\frac{n}{4}$ MaxFlow between $(v_1, v_3), (v_5, v_7)...$, ..., until the last step where we do $1$ MaxFlow between $(v_1, v_{\frac{n}{2}})$. 

  (At step $i$, we do MaxFlow between $(v_1, v_{1 + 2^{i - 1}}), (v_{1 + 2 \cdot 2^{i - 1}}, v_{1 + 3 \cdot 2^{i - 1}}), (v_{1 + 4 \cdot 2^{i - 1}}, v_{1 + 5 \cdot 2^{i - 1}})...$, and the number of MaxFlow is $\frac{n}{2^i}$)

  We do $\Sigma_{i=1}^{\log_{2}{n}}{\frac{n}{2^i}} = n$ total MaxFlow, and the maximum value among these MaxFlow is the min cut of the entire given graph.
  
  \textbf{Time complexity:} with Ford Fulkerson algorithm\footnote{Whose complexity we analyzed in class}, the complexity of each MaxFlow is $O($ the number of edges $\cdot$ the value of the MaxFlow $)$. As we are given an unweighted graph, the value of the MaxFlow is the number of edges $m$. Thus each MaxFlow is $O(m^2)$, and we do $n$ in total, so the overall complexity is $O(m^2 n)$

  \textbf{Correctness:} we want to show that the $n$ pairs would cover the min cut of the graph $(S, T)$, since if so, the algorithm's correct by \textbf{MaxFlow-min-cut Theorem}.

  Both $S$ and $T$ are non-empty, and the min cut $(S, T)$ will be covered if at least one of our pairs $(s', t')$ satisfies $s' \in S$ and $t' \in T$. Assume that none of our pairs satisfies the condition. Given the description above, we know at step $1$ nodes $(v_1, v_2)$ and nodes $(v_3, v_4)$ are each in the same $S$ or $T$, and in step $2$, we know that nodes $(v_1, v_3)$ are in the same $S$ or $T$, thus $v_1...v_4$ are in the same $S$ or $T$. Similarly for each step, to satisfy our assumption, doing MaxFlow for $(v_1, v_{\frac{k}{2}})$ shows that nodes $v_1...v_k$ are in the same $S$ or $T$. Until the last step we have $v_1...v_n$ are all in the same $S$ or $T$, which contradicts with both being non-empty. 

  Thus the min cut of the graph $(S, T)$ will be covered by at least one pair of nodes among the $n$ MaxFlow, and the algorithm's correct.

\item[2]{Menger's theorem for vertices}
  
  The theorem holds.

  \textbf{Proof:} for the theorem we can consider only directed graphs, as undirected graph can be represented as directed graph with edges both way.

  In the directed graph $G = (V, E)$, for each node $v \in V - {s, t}$, we split $v$ into two nodes $i, o$ connected by an edge $(i, o)$, and all incoming edges to $v$ now goes to $i$, all outgoing edges from $v$ goes from $o$. Each edge has capacity $1$, so that after the split, each $(i, o)$ edge has capacity $1$, and in the MaxFlow each node in the original $G$ would be used at most once. With this we transformed the problem into Menger's theorem for edges \footnote{Which we proved in class}, which is immediate from \textbf{MaxFlow-min-cut Theorem}.

  % https://www.math.ku.edu/~jmartin/courses/math796-S08/notes0331.pdf

\item[3]{Deal cards and select}

  \textbf{Solution:} let undirected unweighted bipartite graph $G = (V, E)$, where $V$ can be divided into two groups \textit{LeftV} and \textit{RightV}, and each group is a part in $G$. 

  For any dealing of cards, let each node in \textit{LeftV} represent a pile of cards, and each node in \textit{RightV} represent a rank. Add an edge between a node $s \in$ \textit{LeftV} and a node $t \in RightV$ for each card of rank $t$ in pile $s$.

  In the bipartite graph $G$, each node $s \in$ \textit{LeftV} has degree $4$, and each node $t \in RightV$ has degree $4$. So for each subset of nodes $S \subseteq$ \textit{LeftV}, $|N(S)| \geq |S|$ (Since otherwise there exists an $S'$, such that degree of $S' = 4 \cdot |S'| > 4 \cdot |N(S')|$, which contradicts with the definition of $N(S')$ since the total degree on $S'$'s side is larger than that on $N(S')$'s side). Thus by \textbf{Frobenius Hall Theorem}\footnote{Which we proved in class}, there exists a perfect matching between \textit{LeftV} and \textit{RightV}. And selecting a card of rank $t$ in pile $s$ only if the edge $(s,t)$ is in the perfect matching shows the conclusion.

  % http://www.math.cmu.edu/~lohp/docs/math/mop2009/graph-theory-more.pdf

\item[4]{Exam scheduling max flow}

  \textbf{Solution:} let undirected unweighted bipartite graph $G = (V, E)$, where $V$ can be divided into two groups \textit{LeftV} and \textit{RightV}, and each group is a part in $G$. 

  For any classes, rooms and times combination, let each node $s_i \in$ \textit{LeftV} represent a class $E_i$, and each node $t_{jk} \in$ \textit{RightV} represent a room $S_j$ at time $T_k$. Add an edge between a node $s_i \in$ \textit{LeftV} and a node $t_jk \in RightV$ only if $E_i < S_j$.

  The maximum number of exams that can be scheduled is the max matching (of edges that share no vertex) in bipartite graph $G$. To solve the max matching in bipartite problem\footnote{Which we talked about in class}, we add a node $a$ that connects to all $s_i \in$ \textit{LeftV}, and a node $b$ that connects to all $t_jk \in$ \textit{RightV}, set the capacity of the newly added edges to $1$, the capacity of the already existing edges to $\infty$, and do MaxFlow from $a$ to $b$.

  A schedule exists iff the result of max flow equals with the number of classes; For each edge $(s_i, t_{jk})$ in the max matching, we assign the class $E_i$ to room $S_j$ at time $T_k$.

\end{description}

\end{document}
