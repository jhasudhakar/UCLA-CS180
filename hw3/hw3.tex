% Author: Zhehao Wang 404380075 zhehao@cs.ucla.edu

% Grammar package: http://tex.stackexchange.com/questions/24886/which-package-can-be-used-to-write-bnf-grammars

\documentclass{article}
\topmargin = 0in
\oddsidemargin = 0in
\evensidemargin = \oddsidemargin
\textwidth = 6.5in
\textheight = 8in
\usepackage{amsthm}
\usepackage{amsmath}
\usepackage{syntax}
\usepackage{graphicx}

\usepackage{algorithm}
\usepackage[noend]{algpseudocode}

\makeatletter
\def\BState{\State\hskip-\ALG@thistlm}
\makeatother

\title{CS180 Homework 3}
\author{Zhehao Wang 404380075 (Dis 1B)}
\date{Apr 16, 2016}

\begin{document}
\maketitle

\begin{description}

\item[1]{Strongly connected components in a directed graph}
  
  (a) Prove SCC graph is a DAG.

  \textbf{Proof by contradiction:} assume that a path $s_i, ..., s_j, s_i$ exists in the SCC graph, where $s_k$ ($i \leq k \leq j$) are the newly created distinct SCC nodes.

  Consider a node $p_i$ in SCC node $s_i$, there exists a path from $p_i$ to any node $p_{i+1}$ in SCC $s_{i+1}$, since there exists a node $q_i$ in SCC $s_i$ that has a directed edge to a node $q_{i+i}$ in SCC $s_{i+1}$, and there exists a path from $p_i$ to $q_i$ and from $q_{i+1}$ to $p_{i+1}$.

  Starting from $p_i$, apply the above conclusion repeatedly till we reach a node $p_j$ in SCC $s_j$, and we have that there exists a path from $p_i$ to $p_j$. Similarly, we have there exists a path from $p_j$ to $p_i$. Thus the nodes $p_i$ and $p_j$ should belong to the same SCC node which is the combination of SCC nodes $s_i$ and $s_j$. This contradicts with the nodes being distinct in the assumption. Thus we have the directed SCC graph is acyclic, which makes it a DAG by definition.

  (b) The algorithm is given in alg \ref{alg:scc-graph}. 

  It follows the hint: we start by calling $getSCC(G)$, the algorithm does a DFS starting from the node $v$ who can reach all nodes in $G$, and labels the nodes by \textit{reverse} post-order number (which is why we have smallest labeled node rather than largest as suggested by the hint). The node with smallest label is remembered, and we then start from that node, do another DFS to find the nodes that it can reach, build a SCC with those node, then remove them, and start again with the smallest labeled node. The $DFS$ function takes additional parameters to tell if this run is doing the initial labeling, or this run is getting a SCC and removing it from the original graph, in case of the latter, several additional parameters are added so that the state is kept during each recursive call.

  % This algorithm may not be correct, if we can start from a node that does not have connectivity to some other nodes? 
  % This is apparently prevented by the spec, we can start from a 'v' who can reach all nodes. What if the spec does not give us this 'v'?

  \begin{algorithm}[H]
  \caption{SCC building algorithm}
  \label{alg:scc-graph}
    \begin{algorithmic}[1]
  
    \Function{DFS}{v, do\_label, smallest\_node, node\_remaining, node\_removed, SCC\_graph}
      \State $v.visited \gets true$
      \If {$do\_label = false$}
        \State $node\_remaining.remove(v)$
        \State $node\_removed.add(v)$
        \If {$v = smallest\_node$}
          \State $smallest\_node \gets node\_remaining.nextSmallest()$
        \EndIf
      \EndIf
      \For {$\{i | (v,i) \in E, i.visited = false\}$}
        \If {$do\_label = true$}
          \State $DFS(i, do\_label, smallest\_node, node\_remaining, node\_removed, SCC\_graph)$
        \Else
          \If {$i \in SCC\_graph$}
          \label{line:in}
            \State $SCC\_graph.addEdge(v,i)$
            \label{line:addedge}
          \Else
            \State $DFS(i, do\_label, smallest\_node, node\_remaining, node\_removed, SCC\_graph)$
          \EndIf
        \EndIf
      \EndFor
      \If {$do\_label$}
        \State $id \gets label(v)$
        \If {$id < smallest\_node$}
          \State $smallest\_node \gets v$
        \EndIf
      \EndIf
    \EndFunction

    \Function{getSCC}{G}
      \State $smallest\_node \gets nil$
      \State $DFS(v, true, smallest\_node, G.nodes, [], nil)$

      \State $smallest\_node\_copy \gets smallest\_node.copy()$
      \State $G.resetVisited()$
      \State $SCC\_graph \gets nil$
      \While {$G.node\_count > 0$}
        \State $G.resetVisited()$
        \State $node\_removed \gets []$
        \State $DFS(smallest\_node, false, smallest\_node\_copy, G.nodes, node\_removed, SCC\_graph)$
        \State $smallest\_node \gets smallest\_node\_copy$
        \State $SCC\_graph.addNode(node\_removed)$
      \EndWhile

      \State \Return $SCC\_graph$
    \EndFunction
    
    \end{algorithmic}
  \end{algorithm}

  \textbf{Time complexity:} this algorithm is $O(|E|)$. The initial DFS traversal visits each edge exactly once and is $O(|E|)$. All the subsequent DFSs combined also visits each edge exactly once, and is $O(|E|)$ (This makes the assumption that calls such as deciding if a node's in a SCC (line \ref{line:in}), and adding edges in the resulting SCC graph (line \ref{line:addedge}) are $O(1)$, which can be achieved by hashing). So the overall complexity is $O(|E|)$.

  \textbf{Correctness:} Let $G=(V,E)$ be the original graph, $G'=(V,E')$ and $E=$ \{edges picked by the initial DFS post order traversal of $G$\}. $G'$ is a DAG. The smallest labeled node $r$ from reverse post order traversal is a sink in $G'$. 

  We want to prove that for all $n \in V$, if there exists a path $p_i$ from $r$ to $n$ in $G$, then there exists a path $p_j$ from $n$ to $r$ in $G$. 



\item[2]{Longest path in DAG}

  (a) algorithm is given in alg \ref{alg:longest-path-unweighted-dag}.

  \begin{algorithm}[h]
  \caption{Longest path in an unweighted DAG}
  \label{alg:longest-path-unweighted-dag}
    \begin{algorithmic}[1]
  
    \Function{longestPath}{G}
      \State $nodeCount \gets len(V)$
      \State $sourceNodes \gets []$
      \State $lastNode \gets nil$
      \For {$\{i|i \in V\}$}
        \If {$i.inDegree = 0$}
          \State $sourceNodes.add(i)$
        \EndIf
      \EndFor
      \State $length \gets 0$
      \While {$nodeCount > 0$}
        \State $newSourceNodes \gets []$
        \For {$\{i|i \in sourceNodes\}$}
          \For {$\{v|(i, v) \in E\}$}
            \State $v.inDegree \gets v.inDegree - 1$
            \If {$v.inDegree = 0$}
              \State $newSourceNodes.add(v)$
              \State $v.from \gets i$
            \EndIf
          \EndFor
          \State $G.remove(i)$
          \State $lastNode \gets i$
          \State $nodeCount \gets nodeCount - 1$
        \EndFor
        \State $sourceNodes \gets newSourceNodes$
        \State $length \gets length + 1$
      \EndWhile

      \State $node \gets lastNode$
      \State $path \gets [lastNode]$
      \While {$node.from \neq nil$}
        \State $path.push(node.from)$
        \State $node = node.from$
      \EndWhile
      \State \Return $length$, $path$
    \EndFunction
    
    \end{algorithmic}
  \end{algorithm}

  \textbf{Time complexity:} this iterative algorithm is $O(|E|)$, as it's using an $O(|E|)$ topological sorting algorithm with number of phases it takes remembered. 

  \textbf{Correctness:} this algorithm does a similar thing as topological sort, except that it counts the phases needed to remove all nodes. The correctness is based on the conclusion that length of longest path in a DAG $G$ is 1 + length of longest path in $G'$, where $G'$ is the remaining graph after all sources in $G$ are removed; this conclusion can be proved by induction. 

  And as we find new sources, we update the \textit{from} field of the node to keep track of the nodes on the longest path.

  (b) algorithm is given in alg \ref{alg:longest-path-weighted-dag}. Similar with the idea of problem (a), we base the algorithm on topological sort, and label each node. Each time the algorithm removes a source node $i$, the nodes $j$ that it connects to will have new label $w(j) = max(w(j), l(i,j) + w(i))$. The largest label in the DAG will be returned as the length of the longest path. 

  \begin{algorithm}[h]
  \caption{Longest path in a weighted DAG}
  \label{alg:longest-path-weighted-dag}
    \begin{algorithmic}[1]
  
    \Function{longestPath}{G}
      \State $nodeCount \gets len(V)$
      \State $sourceNodes \gets []$
      \State $maxNode \gets nil$
      \For {$\{i|i \in V\}$}
        \If {$i.inDegree = 0$}
          \State $sourceNodes.add(i)$
          \State $i.label \gets 0$
        \EndIf
      \EndFor
      \State $maxLength \gets 0$
      \While {$nodeCount > 0$}
        \State $newSourceNodes \gets []$
        \For {$\{i|i \in sourceNodes\}$}
          \For {$\{v|(i, v) \in E\}$}
            \State $v.inDegree \gets v.inDegree - 1$
            \If {$v.label < i.label + l(i,v)$}
              \State $v.label \gets i.label + l(i,v)$
              \State $v.from \gets i$
            \EndIf
            \If {$maxLength < v.label$}
              \State {$maxLength \gets v.label$}
              \State {$maxNode \gets v$}
            \EndIf
            \If {$v.inDegree = 0$}
              \State $newSourceNodes.add(v)$
            \EndIf
          \EndFor
          \State $G.remove(i)$
          \State $nodeCount \gets nodeCount - 1$
        \EndFor
        \State $sourceNodes \gets newSourceNodes$
      \EndWhile

      \State $node \gets maxNode$
      \State $path \gets [maxNode]$
      \While {$node.from \neq nil$}
        \State $path.push(node.from)$
        \State $node = node.from$
      \EndWhile
      \State \Return $maxLength$, $path$
    \EndFunction
    
    \end{algorithmic}
  \end{algorithm}

  \textbf{Time complexity:} similar with a topological sort, this algorithm is $O(|E|)$, as each edge in the graph will be used exactly once.

  \textbf{Correctness:} 

  \textbf{Lemma:} during the topological sort, every time a node $p$ becomes source, its label will be the the length of the longest path starting from any node and ending at $p$. 

  This lemma is easily proved by contradiction, as when the node becomes source there won't be any path going to it, so the label of the source will be the length of maximum path that ends at the node. (The labels of all nodes are initialized as 0, so that negative labels won't be kept, and in case of negative labels, we could just ignore the earlier paths that lead to this node.)

  The algorithm returns the maximum label of nodes in the graph after all nodes become sources, given the lemma, the algorithm will return the length of the longest path in the graph. And as we update the label with larger values, we update the \textit{from} field of the node to keep track of the longest nodes on the path.

  (c) algorithm is given in alg \ref{alg:weighted-dag-job-scheduling}. Similar idea as in (b). The algorithm returns a list of jobs sorted by their labels (defined in the algorithm as the longest time it takes to get the prerequisites of that job done), and a schedule that starts each job at the labeled time minimizes the total time needed to finish all jobs.

  \begin{algorithm}[h]
  \caption{Weighted DAG job scheduling}
  \label{alg:weighted-dag-job-scheduling}
    \begin{algorithmic}[1]
  
    \Function{longestPath}{G}
      \State $nodeCount \gets len(V)$
      \State $sourceNodes \gets []$
      \For {$\{i|i \in V\}$}
        \If {$i.inDegree = 0$}
          \State $sourceNodes.add(i)$
          \State $i.label \gets 0$
        \EndIf
      \EndFor
      \While {$nodeCount > 0$}
        \State $newSourceNodes \gets []$
        \For {$\{i|i \in sourceNodes\}$}
          \For {$\{v|(i, v) \in E\}$}
            \State $v.inDegree \gets v.inDegree - 1$
            \State $v.label \gets max(v.label, i.label + l(i))$
            
            \If {$v.inDegree = 0$}
              \State $newSourceNodes.add(v)$
            \EndIf
          \EndFor
          \State $G.remove(i)$
          \State $nodeCount \gets nodeCount - 1$
        \EndFor
        \State $sourceNodes \gets newSourceNodes$
      \EndWhile
      \State \Return $sort(v.label)$
    \EndFunction
    
    \end{algorithmic}
  \end{algorithm}

  \textbf{Time complexity:} this algorithm is $O(max(|E|, |V| \log |V|))$. $O(|E|)$ comes from the topological sort and labeling, which will visit each edge exactly once; and $|V| \log |V|$ comes from the sorting by label in the end.

  \textbf{Correctness:} the proof is similar as that of (b) except that we don't need to consider the negative label case. A brief description below:

  \textbf{Lemma:} during the topological sort, when a node becomes source, its label is the earliest time when the corresponding job can start. This can be easily proved by induction given the label definition in alg \ref{alg:weighted-dag-job-scheduling}.

  By the lemma, we have that the algorithm provides a schedule that starts a job as soon as it can be started. This schedule is optimal, which can be easily proved by contradiction.

\item[3]{Optimal order of files}
  
  Algorithm is given in alg \ref{alg:optimal-order-of-files}, which is a greedy algorithm that orders files as $f_{t_1}...f_{t_n}$, such that for each $t_i > t_j$, $\frac{p_{t_i}}{l_{t_i}} \geq \frac{p_{t_j}}{l_{t_j}}$.

  \begin{algorithm}[h]
  \caption{Optimal order of files}
  \label{alg:optimal-order-of-files}
    \begin{algorithmic}[1]
  
    \Function{optimalOrder}{files}
      \State $weightedFiles \gets []$
      \For {$\{i|i \in files\}$}
        \State $weightedFiles.add(\frac{p_i}{l_i})$
      \EndFor
      \State \Return $sort(weightedFiles)$
    \EndFunction
    
    \end{algorithmic}
  \end{algorithm}

  \textbf{Time complexity:} this algorithm is $O(n \log n)$, where $n$ is the number of files. The calculation of $\frac{p_i}{l_i}$ requires a traversal of array, which is $O(n)$, and the sorting afterwards is $O(n \log n)$, which makes the entire algorithm $O(n \log n)$.

  \textbf{Correctness:} proof by the algorithm's greedy choice property and optimal substructure property.

  \textbf{Greedy choice property:} there exists an optimal solution $S=f_{t_1}...f_{t_n}$, such that $\frac{p_{t_1}}{l_{t_1}}$ is the largest among all jobs.

  % The assumed optimal solution takes a specific form, is this Ok in the proof; if not, is Ang's proof for maximum contatenation correct in dis 3? The idea of that one is bubble sorting, should be better described in the notes as well
  Proof: suppose $S'$ is an optimal solution with the sequence $f_{t'_1}f_{t'_2}...f_{t_{k-1}}f_{t_1}f_{t_{k+1}}...f_{t_n}$ where $t'_1 \neq t_1$. The average access time of $S'$ is 

  $$t(S') = \sum_{i=2}^{n}{((\sum_{j=1}^{i-1}{l_{t'_j}}) \cdot p_{t'_i})}$$

  Consider the solution $S''$ with $f_{t_1}$ and $f_{t_{k-1}}$ swapped, the difference between average access time of $S''$ and that of $S'$ is

  $$t(S') - t(S'') = p_{t_{1}} * l_{t_{k-1}} - p_{t_{k-1}} * l_{t_1} = (\frac{p_{t_1}}{l_{t_1}} - \frac{p_{t_{k-1}}}{l_{t_{k-1}}}) \cdot l_{t_1} \cdot l_{t_{k-1}} \geq 0$$

  Similarly, starting from $t(S'')$, each time we swap $f_{t_1}$ with its previous file, we'll have a smaller or equal access time than before. Thus the optimal solution should contain $f_{t_1}$ as its first element, whose $\frac{p_{t_1}}{l_{t_1}}$ is the largest.

  \textbf{Optimal substructure property:} let $S=f_{t_1}...f_{t_n}$ be an optimal solution, then $S_1=f_{t_2}...f_{t_n}$ is the optimal solution for the sub problem without $f_{t_1}$.

  Proof by contradiction: assume there's a better solution $S'_1=f_{t''_2}...f_{t''_n}$, $t(S'_1)<t(S_1)$ for the sub problem without $f_{t_1}$. Then $S'=f_{t_1}S'_1$ has the average access time of $t(S')=t(S'_1) + l_{t_1} * (1-p_{t_1}) < t(S_1) + l_{t_1} * (1-p_{t_1}) = t(S)$, which contradicts with $S$ being the optimal solution for the original problem.

  With both properties, the greedy algorithm in question is correct.

\item[4]{Sorting from SC}
  
  Suppose that $SC(p_i, d_i)$ takes in $n$ jobs, and each has a $p_i$ and $d_i$. $SC$ returns the optimal list of jobs expressed in two arrays $times$, $deadlines$. The sorting algorithm is described in alg \ref{alg:sorting-sc}.

  \begin{algorithm}[h]
  \caption{Sorting using SC}
  \label{alg:sorting-sc}
    \begin{algorithmic}[1]
  
    \Function{sort}{array}
      \State $times, deadlines \gets SC(array, array)$
      \State \Return $deadlines$
    \EndFunction
    
    \end{algorithmic}
  \end{algorithm}

  \textbf{Time complexity:} SC is $o(n \log n)$, and other calls are $O(1)$, so the overall $sort$ is $o(n \log n)$.

  \textbf{Correctness:} suppose that the algorithm gives back a deadline array of $S=d_1...d_n$, we want to prove for each $i<j$, $d_i \leq d_j$.

  We start by proving that the first element $d_1$ is a smallest element by contradiction. 

  Assume we have a smaller element $d_k < d_1$ in $S$, $1 \leq i \leq n$. Let the maximum lateness of $S$ be $t(S) = max((\sigma_{i=1}^{n}{p_i}) - d_n)$. Consider the sequence $S'$ with $d_k$ and $d_{k-1}$ swapped, let its maximum lateness be $t(S')$. With given input where $p_i = d_i$, $t(S) = \sigma_{i=1}^{n-1}{d_i}$, $t(S') = $, and either way $t(S') \leq t(S)$. 

  Similarly, starting from $S'$, we repeatedly switch $d_k$ with its previous element, and can prove $d_1 =...= d_{k-1} = d_k$. Thus $d_1$ is a smallest element.

  Consider the array $S_1$ which is $S$ with $d_1$ removed, using the above described process we can prove that $d_2$ is a smallest element in $S_1$. We continue until the array $S$ is exhausted, and we have for each $i<j$, $d_i \leq d_j$.

\end{description}

\end{document}
