% Author: Zhehao Wang 404380075 zhehao@cs.ucla.edu

% Thanks to Haitao Zhang for helping with (trying to) catch up with the class, and with the latex template
% Grammar package: http://tex.stackexchange.com/questions/24886/which-package-can-be-used-to-write-bnf-grammars

\documentclass{article}
\topmargin = 0in
\oddsidemargin = 0in
\evensidemargin = \oddsidemargin
\textwidth = 6.5in
\textheight = 8in
\usepackage{bcprules}
\usepackage{amsthm}
\usepackage{amsmath}
\usepackage{syntax}

\usepackage{algorithm}
\usepackage[noend]{algpseudocode}

\makeatletter
\def\BState{\State\hskip-\ALG@thistlm}
\makeatother

\title{Homework 1}
\author{Zhehao Wang 404380075 (Dis 1D)}
\date{Apr 4, 2016}

\begin{document}
\maketitle

\begin{description}

\item[1]{Binary addition algorithm correctness proof}
  
  The input number $n$ can be denoted as $n=a_k...a_0$ in binary, where $a_i=0$, or $a_i=1$ ($0$ $\leq$ $i$ $\leq$ $k$, $k$ is the most significant bit). The flipped number $n'$ can be denoted as  $n'=a'_k...a'_0$ in binary.
  We have 
  \[
  n = \sum_{j=0}^{k}{a_j \cdot 2^j}
  \qquad \text{and} \qquad 
  n' = \sum_{j=0}^{k}{a'_j \cdot 2^j}
  \]
  
  Denote the position of first $0$ in $n$ from right to left to $i$, we have 
  $$n = \sum_{j=0}^{i-1}{1 \cdot 2^j} + 0 \cdot 2^i + \sum_{j=i+1}^{k}{a_j \cdot 2^j} = \frac{2^i-1}{2-1} + \sum_{j=i+1}^{k}{a_j \cdot 2^j} = 2^i - 1 + \sum_{j=i+1}^{k}{a_j \cdot 2^j}$$ 
  and
  $$n + 1 = 2^i + \sum_{j=i+1}^{k}{a_j \cdot 2^j}$$

  After the flip in question, resulting number $n'$ can be denoted as
  $$n' = \sum_{j=0}^{k}{a'_j \cdot 2^j} = \sum_{j=0}^{i-1}{0 \cdot 2^j} + 1 \cdot 2^i + \sum_{j=i+1}^{k}{a'_j \cdot 2^j} = 2^i + \sum_{j=i+1}^{k}{a_j \cdot 2^j}$$

  Thus we have $n' = n + 1$, and the binary addition algorithm in question is correct.

\item[2]{Binary tree depth algorithm}

  \begin{algorithm}
  \caption{Binary tree depth recursive}
    \begin{algorithmic}[1]
    \Function{maxDepth}{node}
      \If {$node is nil$} 
        \Return 0
      \EndIf

      \State $leftDepth \gets maxDepth(node.leftChild)$.
      \State $rightDepth \gets maxDepth(node.rightChild)$.

      \If {$leftDepth > rightDepth$} 
        \Return $leftDepth + 1$
      \Else {} 
        \Return $rightDepth + 1$
      \EndIf
    \EndFunction
    \end{algorithmic}
  \end{algorithm}

  O(n) with the number of nodes in the tree, as each node will be visited exactly once.

\item[3]{Elementary-school-division algorithm}

\item[4]{NIM game}

(a)

(b)

\end{description}

\end{document}
